This paper examines the empirical significance of learning, a type of adaptive, boundedly rational expectations, in the U.S. economy within the framework of the New Keynesian model with endogenous capital accumulation.  Estimation results for learning models can be sensitive to the choice for agents' initial expectations, so three methods for choosing initial expectations are examined.  Maximum likelihood results show that learning under all methods do not significantly improve the fit the model.  The evolution of forecast errors show that the learning models do not out perform the rational expectations model during the run-up of inflation in the 1970s and the subsequent decline in the 1980s, a period of U.S. history which others have suggested learning may play a role.  Despite the failure of learning models to better explain the data, analysis of the impulse response functions and paths of structural shocks during the sample show that learning can lead to different explanations for the data.